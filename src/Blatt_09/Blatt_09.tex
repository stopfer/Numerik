%Schriftgröße, Layout, Papierformat, Art des Dokumentes
\documentclass[10pt,oneside,a4paper]{scrartcl}

%Einstellungen der Seitenränder
\usepackage[left=3cm,right=4cm,top=3cm,bottom=3cm,includeheadfoot]{geometry}

%neue Rechtschreibung
\usepackage{ngerman}

%Umlaute ermöglichen
\usepackage[utf8]{inputenc}
\usepackage[T1]{fontenc}

%Kopf- und Fußzeile
\usepackage{fancyhdr}
\pagestyle{fancy}
\fancyhf{}

%ermöglicht Text ein- / auszublenden
\usepackage{comment}

%\includecomment{Vortrag}
\excludecomment{Vortrag}

%to do
\usepackage{todonotes}

%Mathesachen
\usepackage{amsmath}
\usepackage{amssymb}

%Zeichnen
\usepackage{tikz}

%ermöglicht, Abbildung nicht zu benennen
\usepackage{caption}

%Linie oben
\renewcommand{\headrulewidth}{0.5pt}

%Linie unten
\renewcommand{\footrulewidth}{0.5pt}

%keine Einrückung am Absatzanfang
\parindent=0pt

%for captionof{figure}{"caption"}
\usepackage{caption}

%%for c++ syntax highlights
\usepackage{listings} 
\usepackage{verbatim}

%%for tikz
\usepackage{tikz,times}
\usetikzlibrary{mindmap,trees,backgrounds,arrows}
\tikzstyle{every picture}+=[remember picture]
\everymath{\displaystyle}

%%some colors
\definecolor{myblue}{RGB}{80,80,160}
\definecolor{mygreen}{RGB}{80,160,80}
\definecolor{darkblue}{rgb}{0,0,.6}
\definecolor{darkred}{rgb}{.6,0,0}
\definecolor{darkgreen}{rgb}{0,.5,0}
\definecolor{red}{rgb}{.98,0,0}
\definecolor{shellbackgroundcolor}{rgb}{1,1,1}
\definecolor{shellfontcolor}{rgb}{0,0,0}
\definecolor{shadethmcolor}{rgb}{1,1,1}
\definecolor{bg}{rgb}{0.975,0.98,1}
\definecolor{shaderulecolor}{rgb}{0.40,0.45,0.40}%
\definecolor{mywhite}{rgb}{1,1,1}
\definecolor{myblack}{rgb}{0,0,0}

%%c++ newenvironment
\lstset{%
  language=C++,
  basicstyle= \fontsize{9}{1} \ttfamily,
  commentstyle=\itshape\color{darkgreen},
  keywordstyle=\bfseries\color{darkblue},
  stringstyle=\color{darkred},
  showspaces=false,
  showtabs=false,
  columns=fixed,
  %numbers=left,
  tabsize=3,
  %frame=trBL,
  backgroundcolor=\color{white},
  rulecolor=\color{shaderulecolor},
  %frame=single ,
  numberstyle=\tiny,
  breaklines=true,
  showstringspaces=false,
  xleftmargin=0cm,
  nolol=true,
  captionpos=b,
  morekeywords={Vector,VectorVector,EE,DglE1} }

\lstnewenvironment{cppcode}[1][]%
{
%\begin{center}
\minipage{\textwidth}
\renewcommand\lstlistingname{C++ Code Snipped}%
\renewcommand\lstlistlistingname{c++ Code Snippeds}%
\lstset{%
  language=C++,
  basicstyle= \fontsize{9}{9} \ttfamily,
  commentstyle=\bfseries\itshape\color{darkgreen},
  keywordstyle=\bfseries\color{darkblue},
  stringstyle=\color{darkred},
  showspaces=false,
  showtabs=false,
  columns=fixed,
  %numbers=left,
  tabsize=3,
  caption=#1,
  frame=tb,
  backgroundcolor=\color{bg},
  rulecolor=\color{black},
  %frame=single ,
  %numberstyle=\tiny,
  breaklines=true,
  showstringspaces=false,
  xleftmargin=0cm,
  nolol=true,
  captionpos=b,
  morekeywords={VectorVector,DglE1,EE,Model,Solver} }
}
{
\endminipage
%\end{center}
}

\newcommand{\parspace}{ $\;$\\  \\ }

\newcommand{\code}[1]{\lstinline| #1 |}

\begin{document}
    \begin{center}
        \huge % Schriftgröße einstellen
        \bfseries % Fettdruck einschalten
        \sffamily % Serifenlose Schrift
        Numerik Blatt 9\\[1em]
        \normalsize
        Kathrin Ronellenfitsch, Thorsten Beier, Christopher Pommrenke
    \end{center}

    \section*{Aufgabe 1}
    \subsection*{1.}
    Es gilt:
    \begin{align*}
        \rho(\lambda) &= \lambda^2 - 1 \\
        \sigma(\lambda) &= 2\lambda \\
        \pi(\lambda; \overline{h}) &= \rho(\lambda) -
        \overline{h}\sigma(\lambda) = \lambda^2 - 2\overline{h}\lambda - 1
    \end{align*}
    Für die Nullstellen von $\pi(\lambda; \overline{h})$ gilt:
    \begin{align*}
        \lambda_1 & = \overline{h} - \sqrt{\overline{h}^2 + 1} \\
        \lambda_2 & = \overline{h} + \sqrt{\overline{h}^2 + 1}
    \end{align*}
    Und somit:
        \begin{align*}
        |\lambda_1| &\leq 1 \text{ für } \overline{h} \in [0, \infty) \\
        |\lambda_2| &\leq 1 \text{ für } \overline{h} \in (-\infty, 0]
    \end{align*}
    Daher gilt für das Stabilitätsintervall:
    \begin{equation*}
        \text{SI} = [0, 0]
    \end{equation*}
        
    \subsection*{2.}
    Es gilt:
    \begin{align*}
        \rho(\lambda) &= \lambda^2 - 1 \\
        \sigma(\lambda) &= \frac{1}{2}(\lambda + 3) \\
        \pi(\lambda; \overline{h}) &= \rho(\lambda) -
        \overline{h}\sigma(\lambda) = \lambda^2 - \frac{1}{2}\overline{h}\lambda
        - \frac{3}{2}\overline{h} - 1
    \end{align*}
    Für die Nullstellen von $\pi(\lambda; \overline{h})$ gilt:
    \begin{align*}
        \lambda_1 & = \frac{\overline{h} - \sqrt{\overline{h}^2 +
        24\overline{h} + 16}}{4} \\
        \lambda_2 & = \frac{\overline{h} + \sqrt{\overline{h}^2 +
        24\overline{h} + 16}}{4}
    \end{align*}
    Und somit:
        \begin{align*}
        |\lambda_1| &\leq 1 \text{ für } \overline{h} \in [4(-3 + 2 \sqrt{2}),
        0]
        \\
        |\lambda_2| &\leq 1 \text{ für } \overline{h} \in [4(-3 + 2 \sqrt{2}),
        0]
    \end{align*}
    Daher gilt für das Stabilitätsintervall:
    \begin{equation*}
        \text{SI} = [4(-3 + 2 \sqrt{2}), 0]
    \end{equation*}
    
    \section*{Aufgabe 3}
    Man erhält:
    \begin{equation*}
        u(t = 34500) = u(t = 35000) = u(t = 35500) =  
        \begin{pmatrix}
            2.14159\\
            3.14159\\
        \end{pmatrix}
    \end{equation*}
    \begin{equation*}
        \Rightarrow
        \lim\limits_{t \rightarrow \infty}{u(t)} = 
        \begin{pmatrix}
            2.14159\\
            3.14159\\
        \end{pmatrix}
    \end{equation*}
\end{document}