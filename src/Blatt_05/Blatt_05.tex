%Schriftgröße, Layout, Papierformat, Art des Dokumentes
\documentclass[10pt,oneside,a4paper]{scrartcl}

%Einstellungen der Seitenränder
\usepackage[left=3cm,right=4cm,top=3cm,bottom=3cm,includeheadfoot]{geometry}

%neue Rechtschreibung
\usepackage{ngerman}

%Umlaute ermöglichen
\usepackage[utf8]{inputenc}
\usepackage[T1]{fontenc}

%Kopf- und Fußzeile
\usepackage{fancyhdr}
\pagestyle{fancy}
\fancyhf{}

%ermöglicht Text ein- / auszublenden
\usepackage{comment}

%\includecomment{Vortrag}
\excludecomment{Vortrag}

%to do
\usepackage{todonotes}

%Mathesachen
\usepackage{amsmath}
\usepackage{amssymb}

%Zeichnen
\usepackage{tikz}

%ermöglicht, Abbildung nicht zu benennen
\usepackage{caption}

%Linie oben
\renewcommand{\headrulewidth}{0.5pt}

%Linie unten
\renewcommand{\footrulewidth}{0.5pt}

%keine Einrückung am Absatzanfang
\parindent=0pt

%for captionof{figure}{"caption"}
\usepackage{caption}

%%for c++ syntax highlights
\usepackage{listings} 
\usepackage{verbatim}

%%for tikz
\usepackage{tikz,times}
\usetikzlibrary{mindmap,trees,backgrounds,arrows}
\tikzstyle{every picture}+=[remember picture]
\everymath{\displaystyle}

%%some colors
\definecolor{myblue}{RGB}{80,80,160}
\definecolor{mygreen}{RGB}{80,160,80}
\definecolor{darkblue}{rgb}{0,0,.6}
\definecolor{darkred}{rgb}{.6,0,0}
\definecolor{darkgreen}{rgb}{0,.5,0}
\definecolor{red}{rgb}{.98,0,0}
\definecolor{shellbackgroundcolor}{rgb}{1,1,1}
\definecolor{shellfontcolor}{rgb}{0,0,0}
\definecolor{shadethmcolor}{rgb}{1,1,1}
\definecolor{bg}{rgb}{0.975,0.98,1}
\definecolor{shaderulecolor}{rgb}{0.40,0.45,0.40}%
\definecolor{mywhite}{rgb}{1,1,1}
\definecolor{myblack}{rgb}{0,0,0}

%%c++ newenvironment
\lstset{%
  language=C++,
  basicstyle= \fontsize{9}{1} \ttfamily,
  commentstyle=\itshape\color{darkgreen},
  keywordstyle=\bfseries\color{darkblue},
  stringstyle=\color{darkred},
  showspaces=false,
  showtabs=false,
  columns=fixed,
  %numbers=left,
  tabsize=3,
  %frame=trBL,
  backgroundcolor=\color{white},
  rulecolor=\color{shaderulecolor},
  %frame=single ,
  numberstyle=\tiny,
  breaklines=true,
  showstringspaces=false,
  xleftmargin=0cm,
  nolol=true,
  captionpos=b,
  morekeywords={Vector,VectorVector,EE,DglE1} }

\lstnewenvironment{cppcode}[1][]%
{
%\begin{center}
\minipage{\textwidth}
\renewcommand\lstlistingname{C++ Code Snipped}%
\renewcommand\lstlistlistingname{c++ Code Snippeds}%
\lstset{%
  language=C++,
  basicstyle= \fontsize{9}{9} \ttfamily,
  commentstyle=\bfseries\itshape\color{darkgreen},
  keywordstyle=\bfseries\color{darkblue},
  stringstyle=\color{darkred},
  showspaces=false,
  showtabs=false,
  columns=fixed,
  %numbers=left,
  tabsize=3,
  caption=#1,
  frame=tb,
  backgroundcolor=\color{bg},
  rulecolor=\color{black},
  %frame=single ,
  %numberstyle=\tiny,
  breaklines=true,
  showstringspaces=false,
  xleftmargin=0cm,
  nolol=true,
  captionpos=b,
  morekeywords={VectorVector,DglE1,EE,Model,Solver} }
}
{
\endminipage
%\end{center}
}

\newcommand{\parspace}{ $\;$\\  \\ }

\newcommand{\code}[1]{\lstinline| #1 |}

\begin{document}
    \begin{center}
        \huge % Schriftgröße einstellen
        \bfseries % Fettdruck einschalten
        \sffamily % Serifenlose Schrift
        Numerik Blatt 5\\[1em]
        \normalsize
        Kathrin Ronellenfitsch, Thorsten Beier, Christopher Pommrenke
    \end{center}

    
    \section*{Aufgabe 1}
        Für die skalare Testgleichung
        \begin{equation*}
            u'(t) = \lambda * u(t), \quad \lambda \in{\mathbb{C}}
        \end{equation*}
        gilt:
        
        \subsection*{a)}
            \begin{align*}
                y_{n + 1} &= y_n + \frac{1}{2} * h * (\lambda * y_{n + 1} +
                \lambda * y_n)\\
                \Rightarrow\\
                y_{n + 1} &= \frac{1 + \frac{1}{2} * h * \lambda}{1 -
                \frac{1}{2} * h * \lambda} * y_n\\
                &= (\frac{-4}{h * \lambda - 2} - 1) * y_n\\
                \Rightarrow\\
                w(z) &= \frac{-4}{z - 2} - 1 \quad \Rightarrow z \leq 0\\
                &\Rightarrow \text{SI} = (-\infty, 0]                
            \end{align*}

        \subsection*{b)}
            \begin{align*}
                y_{n + 1} &= y_n + h * \lambda * (y_n + \frac{1}{2} * h *
                \lambda * y_n)\\
                \Rightarrow\\
                y_{n + 1} &= y_n * (1 + h * \lambda + \frac{1}{2} * (h *
                \lambda)^2)\\
                \Rightarrow\\ w(z) &= 1 + z + \frac{z^2}{2} \quad
                \Rightarrow -2 \leq z \leq 0\\ &\Rightarrow \text{SI} =
                [-2, 0]
            \end{align*}
        \subsection*{c)}
    
    \section*{Aufgabe 2}
       Für ein Runge-Kutta Verfahren der Ordnung R gilt:
       \begin{equation*}
           w(z) = \sum\limits_{r=0}^R \frac{z^r}{r!}
       \end{equation*}
       Mit $\Re(z) = 0$ gilt:
       \subsection*{R = 1}
           \begin{align*}
               w(z) &= \sum\limits_{r=0}^1 \frac{z^r}{r!}
               &= 1 + z\\
               |w(z)| &= 1^2 + z^2 \leq 1 \qquad \Rightarrow  &z \stackrel{!}{=}
               0\\
               &\Rightarrow \imath\text{SI} = [0, 0]
           \end{align*}
      \subsection*{R = 2}
           \begin{align*}
               w(z) &= \sum\limits_{r=0}^2 \frac{z^r}{r!}
               &= 1 + z + \frac{z^2}{2}\\
              |w(z)| &= (1 - \frac{z^2}{2})^2 + z^2 = 1 + \frac{z^4}{4} \leq 1
              \qquad \Rightarrow  &z \stackrel{!}{=} 0\\
               &\Rightarrow \imath\text{SI} = [0, 0]
           \end{align*}
      \subsection*{R = 3}
           \begin{align*}
               w(z) &= \sum\limits_{r=3}^2 \frac{z^r}{r!}
               &= 1 + z + \frac{z^2}{2} + \frac{z^3}{6}\\
              |w(z)| &= (1 - \frac{z^2}{2})^2 + (z - \frac{z^3}{6})^2 = 1 +
              \frac{z^6}{36} - \frac{z^4}{12} \leq 1 \qquad \Rightarrow  &z^2
              \leq 3\\ &\Rightarrow \imath\text{SI} = [-\sqrt{3}, \sqrt{3}]
           \end{align*}
\end{document}
