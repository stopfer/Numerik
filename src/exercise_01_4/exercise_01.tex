%Schriftgröße, Layout, Papierformat, Art des Dokumentes
\documentclass[10pt,oneside,a4paper]{scrartcl}

%Einstellungen der Seitenränder
\usepackage[left=3cm,right=4cm,top=3cm,bottom=3cm,includeheadfoot]{geometry}

%neue Rechtschreibung
\usepackage{ngerman}

%Umlaute ermöglichen
\usepackage[latin1]{inputenc}

%Kopf- und Fußzeile
\usepackage{fancyhdr}
\pagestyle{fancy}
\fancyhf{}

%ermöglicht Text ein- / auszublenden
\usepackage{comment}

%\includecomment{Vortrag}
\excludecomment{Vortrag}

%to do
\usepackage{todonotes}

%Mathesachen
\usepackage{amsmath}

%Zeichnen
\usepackage{tikz}

%ermöglicht, Abbildung nicht zu benennen
\usepackage{caption}

%Linie oben
\renewcommand{\headrulewidth}{0.5pt}

%Linie unten
\renewcommand{\footrulewidth}{0.5pt}

%for captionof{figure}{"caption"}
\usepackage{caption}

%%for c++ syntax highlights
\usepackage{listings} 
\usepackage{verbatim}

%%for tikz
\usepackage{tikz,times}
\usetikzlibrary{mindmap,trees,backgrounds,arrows}
\tikzstyle{every picture}+=[remember picture]
\everymath{\displaystyle}

%%some colors
\definecolor{myblue}{RGB}{80,80,160}
\definecolor{mygreen}{RGB}{80,160,80}
\definecolor{darkblue}{rgb}{0,0,.6}
\definecolor{darkred}{rgb}{.6,0,0}
\definecolor{darkgreen}{rgb}{0,.5,0}
\definecolor{red}{rgb}{.98,0,0}
\definecolor{shellbackgroundcolor}{rgb}{1,1,1}
\definecolor{shellfontcolor}{rgb}{0,0,0}
\definecolor{shadethmcolor}{rgb}{1,1,1}
\definecolor{bg}{rgb}{0.975,0.98,1}
\definecolor{shaderulecolor}{rgb}{0.40,0.45,0.40}%
\definecolor{mywhite}{rgb}{1,1,1}
\definecolor{myblack}{rgb}{0,0,0}

%%c++ newenvironment
\lstset{%
  language=C++,
  basicstyle= \fontsize{9}{1} \ttfamily,
  commentstyle=\itshape\color{darkgreen},
  keywordstyle=\bfseries\color{darkblue},
  stringstyle=\color{darkred},
  showspaces=false,
  showtabs=false,
  columns=fixed,
  %numbers=left,
  tabsize=3,
  %frame=trBL,
  backgroundcolor=\color{white},
  rulecolor=\color{shaderulecolor},
  %frame=single ,
  numberstyle=\tiny,
  breaklines=true,
  showstringspaces=false,
  xleftmargin=0cm,
  nolol=true,
  captionpos=b,
  morekeywords={Vector,VectorVector,EE,DglE1} }

\lstnewenvironment{cppcode}[1][]%
{
%\begin{center}
\minipage{\textwidth}
\renewcommand\lstlistingname{C++ Code Snipped}%
\renewcommand\lstlistlistingname{c++ Code Snippeds}%
\lstset{%
  language=C++,
  basicstyle= \fontsize{9}{9} \ttfamily,
  commentstyle=\bfseries\itshape\color{darkgreen},
  keywordstyle=\bfseries\color{darkblue},
  stringstyle=\color{darkred},
  showspaces=false,
  showtabs=false,
  columns=fixed,
  %numbers=left,
  tabsize=3,
  caption=#1,
  frame=tb,
  backgroundcolor=\color{bg},
  rulecolor=\color{black},
  %frame=single ,
  %numberstyle=\tiny,
  breaklines=true,
  showstringspaces=false,
  xleftmargin=0cm,
  nolol=true,
  captionpos=b,
  morekeywords={VectorVector,DglE1,EE,Model,Solver} }
}
{
\endminipage
%\end{center}
}

\newcommand{\parspace}{ $\;$\\  \\ }

\newcommand{\code}[1]{\lstinline| #1 |}

\begin{document}
	%\title{Strukturen großer Netzwerke}
	%\author{Kathrin Ronellenfitsch}
	%\maketitle

	\begin{center}
		\huge % Schriftgröße einstellen
		\bfseries % Fettdruck einschalten
		\sffamily % Serifenlose Schrift
		Numerik Blatt 1\\[1em]
		\normalsize
		Kathrin Ronellenfitsch, Thorsten Beier, Christopher Pommrenke
	\end{center}

	
	\section*{Aufgabe 1}
    \subsection*{1.1}
    Betrachtet man die verschiedenen Werte von $h$ als verschiedene Stoerungen
    so konennen  $a(h)$ und $a(h/2)$  exponentiell von einander abweichen daher
    ist ein Ansatz mit $h^{\alpha}$ noetig. Ist der exakte Limes $a$ nicht
    bekannt hat man  in $a(h)-a=ch^{\alpha}$ 3 Unbekannte: $\alpha$, $c$ und $h$
    also brauch man mindestens 3 werte fuer $h$ und $a(h)$ um $\alpha$ zu
    bestimmen (3 gleichungen aufstellen, $c$ und $a$ eliminieren und nach
    $\alpha$ aufloesen). Intuitiv scheint der folgende Ansatz auch halbwegs
    sinnig:
    \begin{equation*}
        \hat{\alpha}= \frac{1} { \log{2}} \log{\left( \left| \frac{a(h)-a(h/2)}{a(h/2)-a(h/4)}\right|\right) }
    \end{equation*} 

	\subsection*{1.2}

	\begin{equation*}
		\alpha = \frac{1}{\log(2)}\log \left( \left|\frac{a(h)-a}{a(h/2)-a}\right| \right)\\
	\end{equation*}

	\begin{align*}
		\text {Für die Funktion } a(h) \text { ist } \alpha &= 0,979458145414579 \text { für } h = 2^{-1} \\
		\alpha &= 0,996501297146864 \text { für } h = 2^{-2} \\
		\alpha &= 1,01229546124656 \text { für } h = 2^{-3} \\
		\alpha &= 1,03497195370122 \text { für } h = 2^{-4} \\
		\alpha &= 1,07783959914214 \text { für } h = 2^{-5} \\
		\alpha_{\text{average}} &= 1,0202132913
	\end{align*}

	\begin{align*}
 		\text{ Für die Funktion } b(h) \text { ist } \alpha &= 1,92766610793977 \text { für } h = 2^{-1} \\
		\alpha &= 1,9859616431747 \text { für } h = 2^{-2} \\
		\alpha &= 2,00004270113147 \text { für } h = 2^{-3} \\
		\alpha &= 2,0153154176937 \text { für } h = 2^{-4} \\
		\alpha &= 2,0712330305954 \text { für } h = 2^{-5} \\
		\alpha_{\text{average}} &= 2,0000437801
 	\end{align*}

	\section*{Aufgabe 2}

	\subsection*{2.1}

	\begin{equation*}	
		p(t_{0}) =\frac {ap_0} {bp_0+ (a-bp_0)e^{-a(t_{0}-t_0)} } = 
				\frac {ap_0} {bp_0+ (a-bp_0)e^0 } = 
				\frac {ap_0} {bp_0+ a-bp_0} =
				\frac {ap_0} {a } = p_0 
	\end{equation*}

	\begin{equation*}
		p(t)  =\frac {ap_0} {bp_0+ (a-bp_0)e^{-a(t-t_0)} } =
			 ap_{0} (bp_0+ (a-bp_0)e^{-a(t + t_0) })^{-1} 
	\end{equation*}

	\begin{align*}
		\frac{d}{dt}p(t) & =
			 ap_{0}(-1) (bp_0+( a -bp_0)e^{-a(t + t_0) })^{-2}
			 (- a)(a-bp_0)e^{-a(t + t_0) }\\
			 & =\frac {a^{2}p_0(a-bp_0)e^{-a(t + t_0)}} {(bp_0+( a -bp_0)e^{a(t 				- at_0) })^{2}}
			 =\frac {a^{2}p_0bp_0 + a^{2}p_0(a-bp_0)e^{-a(t + t_0)} - 		 			a^{2}p_0bp_0} {(bp_0+( a -bp_0)e^{a(t- t_0) })^{2}}\\
			& = \frac {a^{2}p_0(bp_0 + (a-bp_0)e^{-a(t + t_0)}}
			 {(bp_0+( a -bp_0)e^{a(t- t_0) })^{2}} - 
				\frac {ba^{2}p_{0}^2} {(bp_0+( a -bp_0)e^{a(t- t_0) })^{2}}\\
			& =a \frac {ap_0} {(bp_0+( a -bp_0)e^{a(t- t_0) })} - 
				b\left(\frac {ap_{0}} {(bp_0+( a -bp_0)e^{a(t- t_0) })}\right)^2\\
			& = ap(t)-bp(t)^2
	\end{align*}

	\subsection*{2.2}

	\begin{equation*}
		\frac{d}{dt}p(t) = bp(t)^2- ap(t)
	\end{equation*}

	$ \frac{d}{dt}p(t) $ beschreibt den Anstieg bzw. den Abfall der Anzahl der Bevölkerung.\\
	Die Spezies ist gefährdet, wenn die Anzahl der Population ab einem bestimmten Zeitpunkt $ t_0 $ dauerhaft fallend ist, wenn also gilt,\\
	dass $ p(t) $ streng monoton fallend ist, d.h.
		\begin{equation*}
			\frac{d}{dt}p(t) < 0 \text{ für alle } t>t_0\Leftrightarrow
		\end{equation*}

		\begin{equation*}
			bp(t)^2- ap(t) < 0 \Leftrightarrow bp(t)^2 <  ap(t) \Leftrightarrow bp(t) < a \Leftrightarrow p(t) < \frac{a}{b}
		\end{equation*}
	(man kann davon ausgehen, dass $p(t) > 0$ ist und man somit durch $p(t)$ teilen kann, denn ansonsten wäre die Population ja schon ausgestorben)
		

	\section*{Aufgabe 3}

	\subsection*{3.1}

	\begin{align*}
		u(t) & = \sum\limits_{i=1}^{n} \zeta_{i}e^{\alpha_{i}t} \quad  \Rightarrow \\
		u'(t) & =  \sum\limits_{i=1}^{n} \alpha_{i}\zeta_{i}e^{\alpha_{i}t} \\
	\end{align*}
    \ldots

	\subsection*{3.2}

    Charakteristisches Polynom:
    \[ x^2 - 10x + 21 \]

    reelle Eigenwerte:   
    \[  3 ,  7  \]

    Eigenvektor zum Eigenwert 3:
    $
        \left(
        \begin{tabular}{c}
            $1$ \\ 
            $1$
        \end{tabular} 
        \right)
    $

    Eigenvektor zum Eigenwert 7:
    $
        \left(
        \begin{tabular}{c}
            $-1$ \\ 
            $1$
        \end{tabular} 
        \right)
    $

    \[ 
        \left(
        \begin{tabular}{c}
            $x_1(t)$ \\ 
            $x_2(t)$
        \end{tabular} 
        \right)
        =C_1 e^{3 \cdot t}
        \left(
        \begin{tabular}{c}
            $1$ \\ 
            $1$
        \end{tabular} 
        \right)
        +
        C_2 e^{7 \cdot t}
        \left(
        \begin{tabular}{c}
            $-1$ \\ 
            $1$
        \end{tabular} 
        \right)
    \]

    Mit 
    $
        x(0)=
        \left(
        \begin{tabular}{c}
            $1$ \\ 
            $3$
        \end{tabular} 
        \right)
    $
    ergibt sich 

    \[ 
        \left(
        \begin{tabular}{c}
            $x_1(t)$ \\ 
            $x_2(t)$
        \end{tabular} 
        \right)
        =2 e^{3 \cdot t}
        \left(
        \begin{tabular}{c}
            $1$ \\ 
            $1$
        \end{tabular} 
        \right)
        + e^{7 \cdot t}
        \left(
        \begin{tabular}{c}
            $-1$ \\ 
            $1$
        \end{tabular} 
        \right)
    \]

    \newpage
    
    \section*{Aufgabe 4}
    \subsection*{a)}
    
    InitialValueProblem:

    \begin{cppcode}
/** @brief Example class for a differential equation model

    The model is

    u'(t) = lambda*t*u^2(t), -3 <= t <= 3, u(t_0) = 1 / 901.

    \tparam T a type representing time values
    \tparam N a type representing states and f-values
*/
template<class T, class N=T>
class InitialValueProblem {
private:
    const N lambda;
    const N u0;
    const N t0;

public:
    /** \brief export size_type */
    typedef std::size_t size_type;

    /** \brief export time_type */
    typedef T time_type;

    /** \brief export number_type */
    typedef N number_type;

    //! constructor stores parameter lambda
    InitialValueProblem (const N& lambda_, const N& u0_, const N& t0_)
        : lambda(lambda_), u0(u0_), t0(t0_)
    {}

    //! return number of componentes for the model
    std::size_t size () const
    {
        return 1;
    }

    //! set initial state including time value
    void initialize (T& t0, hdnum::Vector<N>& x0) const
    {
        t0 = this->t0;
        x0[0] = u0;
    }

    //! model evaluation
    void f (const T& t, const hdnum::Vector<N>& x, hdnum::Vector<N>& result) const
    {
      result[0] = lambda*t*x[0]*x[0];
    }

    //! jacobian evaluation needed for implicit solvers
    void f_x (const T& t, const hdnum::Vector<N>& x, hdnum::DenseMatrix<N>& result) const 
    {
      throw std::string("Jacobian evaluation not implemented!");
    }
};
    \end{cppcode}
    
    \newpage
    
    \subsection*{b)}
    Approximation der analytischen Lösung:
    
    Verwendete Headder:
    
    \begin{cppcode}
#include <iostream>

#include "hdnum.hh"
#include "initial_value_problem.h"
#include "expliciteuler.hh"          //from examplecode of hdnum
    \end{cppcode}
    
    Hilfsfunktion um bestmögliche Annäherung an einen vorgegebenen Zeitschritt
    zu bestimmen:
    
    \begin{cppcode}
// y - eps < x < y + eps ?
bool inRange(const double x, const double y, const double eps) {
    if(x > y - eps) {
        if (x < y + eps) {
            return true;
        } else {
            return false;
        }
    } else {
        return false;
    }
}
    \end{cppcode}
    
    \newpage
    
    main():
    
    \begin{cppcode}
int main() {
    typedef double Number;                          // define a number type

    const Number t0 = -3.0;                         // initial time
    const Number tStep = 0.00001;                       // delta t
    const Number tIntermediate = 1.0;               // state at this time will be printed on console
    const Number tMax = 3.0;                        // end time
    const Number u0 = 1.0/901.0;                    // initial state
    const Number lambda = -200.0;                   // factor for the model

    typedef InitialValueProblem<Number> Model;      // Model type
    Model model(lambda, u0, t0);                    // instantiate model

    typedef ExplicitEuler<Model> Solver;            // Solver type
    Solver solver(model);                           // instantiate solver
    solver.set_dt(tStep);                           // set initial time step

    hdnum::Vector<Number> times;                    // store time values here
    hdnum::Vector<hdnum::Vector<Number> > states;   // store states here
    times.push_back(solver.get_time());             // initial time
    states.push_back(solver.get_state());           // initial state

    while (solver.get_time() <= tMax)               // the time loop
    {
      solver.step();                                // advance model by one time step
      times.push_back(solver.get_time());           // save time
      states.push_back(solver.get_state());         // and state
      if(inRange(tIntermediate, solver.get_time(), tStep/2.0)) {        // print state at intermediate time
          std::cout << "State at time t = " << solver.get_time() << ": " << solver.get_state() << std::endl;
      }
    }

    gnuplot("exercise_01_4.dat",times,states);      // output model result

    return 0;
}
    \end{cppcode}
    
    \subsection*{c)}
    
    \begin{center}
        \includegraphics[scale=0.5,angle={270}]{exercise_01_4.ps}
    \end{center}
\end{document}
