%Schriftgröße, Layout, Papierformat, Art des Dokumentes
\documentclass[10pt,oneside,a4paper]{scrartcl}

%Einstellungen der Seitenränder
\usepackage[left=3cm,right=4cm,top=3cm,bottom=3cm,includeheadfoot]{geometry}

%neue Rechtschreibung
\usepackage{ngerman}

%Umlaute ermöglichen
\usepackage[utf8]{inputenc}
\usepackage[T1]{fontenc}

%Kopf- und Fußzeile
\usepackage{fancyhdr}
\pagestyle{fancy}
\fancyhf{}

%ermöglicht Text ein- / auszublenden
\usepackage{comment}

%\includecomment{Vortrag}
\excludecomment{Vortrag}

%to do
\usepackage{todonotes}

%Mathesachen
\usepackage{amsmath}
\usepackage{amssymb}

%Zeichnen
\usepackage{tikz}

%ermöglicht, Abbildung nicht zu benennen
\usepackage{caption}

%Linie oben
\renewcommand{\headrulewidth}{0.5pt}

%Linie unten
\renewcommand{\footrulewidth}{0.5pt}

%keine Einrückung am Absatzanfang
\parindent=0pt

%for captionof{figure}{"caption"}
\usepackage{caption}

%%for c++ syntax highlights
\usepackage{listings} 
\usepackage{verbatim}

%%for tikz
\usepackage{tikz,times}
\usetikzlibrary{mindmap,trees,backgrounds,arrows}
\tikzstyle{every picture}+=[remember picture]
\everymath{\displaystyle}

%%some colors
\definecolor{myblue}{RGB}{80,80,160}
\definecolor{mygreen}{RGB}{80,160,80}
\definecolor{darkblue}{rgb}{0,0,.6}
\definecolor{darkred}{rgb}{.6,0,0}
\definecolor{darkgreen}{rgb}{0,.5,0}
\definecolor{red}{rgb}{.98,0,0}
\definecolor{shellbackgroundcolor}{rgb}{1,1,1}
\definecolor{shellfontcolor}{rgb}{0,0,0}
\definecolor{shadethmcolor}{rgb}{1,1,1}
\definecolor{bg}{rgb}{0.975,0.98,1}
\definecolor{shaderulecolor}{rgb}{0.40,0.45,0.40}%
\definecolor{mywhite}{rgb}{1,1,1}
\definecolor{myblack}{rgb}{0,0,0}

%%c++ newenvironment
\lstset{%
  language=C++,
  basicstyle= \fontsize{9}{1} \ttfamily,
  commentstyle=\itshape\color{darkgreen},
  keywordstyle=\bfseries\color{darkblue},
  stringstyle=\color{darkred},
  showspaces=false,
  showtabs=false,
  columns=fixed,
  %numbers=left,
  tabsize=3,
  %frame=trBL,
  backgroundcolor=\color{white},
  rulecolor=\color{shaderulecolor},
  %frame=single ,
  numberstyle=\tiny,
  breaklines=true,
  showstringspaces=false,
  xleftmargin=0cm,
  nolol=true,
  captionpos=b,
  morekeywords={Vector,VectorVector,EE,DglE1} }

\lstnewenvironment{cppcode}[1][]%
{
%\begin{center}
\minipage{\textwidth}
\renewcommand\lstlistingname{C++ Code Snipped}%
\renewcommand\lstlistlistingname{c++ Code Snippeds}%
\lstset{%
  language=C++,
  basicstyle= \fontsize{9}{9} \ttfamily,
  commentstyle=\bfseries\itshape\color{darkgreen},
  keywordstyle=\bfseries\color{darkblue},
  stringstyle=\color{darkred},
  showspaces=false,
  showtabs=false,
  columns=fixed,
  %numbers=left,
  tabsize=3,
  caption=#1,
  frame=tb,
  backgroundcolor=\color{bg},
  rulecolor=\color{black},
  %frame=single ,
  %numberstyle=\tiny,
  breaklines=true,
  showstringspaces=false,
  xleftmargin=0cm,
  nolol=true,
  captionpos=b,
  morekeywords={VectorVector,DglE1,EE,Model,Solver} }
}
{
\endminipage
%\end{center}
}

\newcommand{\parspace}{ $\;$\\  \\ }

\newcommand{\code}[1]{\lstinline| #1 |}

\begin{document}
    \begin{center}
        \huge % Schriftgröße einstellen
        \bfseries % Fettdruck einschalten
        \sffamily % Serifenlose Schrift
        Numerik Blatt 6\\[1em]
        \normalsize
        Kathrin Ronellenfitsch, Thorsten Beier, Christopher Pommrenke
    \end{center}

    
    \section*{Aufgabe 1}

    
    \section*{Aufgabe 2}
    Trapezregel:
    \[
        y_1=y_0 + \frac{h}{2}
        (f(x_0,y_0)+f(x_1,y_1))
        =y_0 + h(\frac{1}{2}K_1+\frac{1}{2}K_2)
    \]
    wobei
    $K_1=f(x_0,y_0),K_2=f(x_1,y_1)=f(x_0,y_0+h(\frac{1}{2}K_1+\frac{1}{2}K_2))$
    \newline
    Die Stabilit\"atsfunktion ist somit:
    \[
        y_1=y_0 + \frac{h}{2}
        (\lambda)y_0+\lambda y_1] \hspace{0.5cm} \Rightarrow \hspace{0.5cm}
        y_1=\frac{1+h \lambda/2}{1-h \lambda/2}
    \]
    und somit
    \[
        R(z)=\frac{1+z/2}{1-z/2}
    \]

    Man betrachtet nun die imagin\"are Achse wo $z=iy$

    \[
        \Rightarrow |\frac{1+iy/2}{1-iy/2}| \Longleftrightarrow |1+ iy/2|^2 \leq
        |1- iy/2|^2
    \]

    Da die Ungleichung erfuellt ist wegen
    \[
        |1+ iy/2|^2=|1 - iy/2|^2
    \] 
    Also ist $|R(z)|\leq 1$ auf dem Rand von $C^-$ ,und $R$ ist im inneren
    analytisch und darum foglgt nach dem Maximusprinzip $|R(z)|\leq 1$ f \"ur
    alle $z \in C^-$ Darum ist die  Trapezregel A-Stabil.
    Die Mittelpunktsregel hat genau die selbe Stabilit\"atsfunktion und ist
    somit auch A-Stabil.
    
    \section*{Aufgabe 3}
    \subsection*{a)}
    Implementierung der step-Methode für die Trapezregel:\\
    \begin{cppcode}
    //! do one step
    void step ()
    {
      mpf_set_default_prec(1024);
      // (E_n - 1/2 * h * A) * y_n = y_(n-1) + 1/2 * A * y_(n-1)
      model.f_x(t, u, A);         // get Matrix
      A *= -(1.0/2.0)*dt;         // Set Matrix for lr decomposition
      A += E;
      model.f(t, u, f);           // evaluate model
      u.update((1.0/2.0)*dt,f);   // set Vector for lr decomposition

      // lr decomposition
      hdnum::Vector<number_type> s(model.size());
      hdnum::Array<size_type> p(model.size());
      hdnum::Array<size_type> q(model.size());
      row_equilibrate(A,s);
      lr_fullpivot(A,p,q);
      apply_equilibrate(s,u);
      f = 0.0;
      permute_forward(p,u);
      solveL(A,u,u);
      solveR(A,f,u);
      permute_backward(q,f);

      u = f;            // set new state
      t += dt;          // advance time
    }
    \end{cppcode}
    
    \subsection*{b)}
    Bei einer Schrittweite $\Delta t = 2^{-27}$ wird der Fehler kleiner als
    $10^{-10}$ und man erhält:\\
    \begin{equation*}
        y_n =    
        \begin{pmatrix}
            9.1578194e-03\\
            9.1578194e-03\\
            3.0377315e-13
        \end{pmatrix}
    \end{equation*}
    bei einem zu erwartenden Wert von:\\
        \begin{equation*}
        u_n =    
        \begin{pmatrix}
            9.1578194e-03\\
            9.1578194e-03\\
            -1.5945887e-35
        \end{pmatrix}
    \end{equation*}
\end{document}
