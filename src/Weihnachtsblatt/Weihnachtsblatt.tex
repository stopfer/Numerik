%Schriftgröße, Layout, Papierformat, Art des Dokumentes
\documentclass[10pt,oneside,a4paper]{scrartcl}

%Einstellungen der Seitenränder
\usepackage[left=3cm,right=4cm,top=3cm,bottom=3cm,includeheadfoot]{geometry}

%neue Rechtschreibung
\usepackage{ngerman}

%Umlaute ermöglichen
\usepackage[utf8]{inputenc}
\usepackage[T1]{fontenc}

%Kopf- und Fußzeile
\usepackage{fancyhdr}
\pagestyle{fancy}
\fancyhf{}

%ermöglicht Text ein- / auszublenden
\usepackage{comment}

%\includecomment{Vortrag}
\excludecomment{Vortrag}

%to do
\usepackage{todonotes}

%Mathesachen
\usepackage{amsmath}
\usepackage{amssymb}

%Zeichnen
\usepackage{tikz}

%ermöglicht, Abbildung nicht zu benennen
\usepackage{caption}

%Linie oben
\renewcommand{\headrulewidth}{0.5pt}

%Linie unten
\renewcommand{\footrulewidth}{0.5pt}

%keine Einrückung am Absatzanfang
\parindent=0pt

%for captionof{figure}{"caption"}
\usepackage{caption}

%%for c++ syntax highlights
\usepackage{listings} 
\usepackage{verbatim}

%%for tikz
\usepackage{tikz,times}
\usetikzlibrary{mindmap,trees,backgrounds,arrows}
\tikzstyle{every picture}+=[remember picture]
\everymath{\displaystyle}

%%some colors
\definecolor{myblue}{RGB}{80,80,160}
\definecolor{mygreen}{RGB}{80,160,80}
\definecolor{darkblue}{rgb}{0,0,.6}
\definecolor{darkred}{rgb}{.6,0,0}
\definecolor{darkgreen}{rgb}{0,.5,0}
\definecolor{red}{rgb}{.98,0,0}
\definecolor{shellbackgroundcolor}{rgb}{1,1,1}
\definecolor{shellfontcolor}{rgb}{0,0,0}
\definecolor{shadethmcolor}{rgb}{1,1,1}
\definecolor{bg}{rgb}{0.975,0.98,1}
\definecolor{shaderulecolor}{rgb}{0.40,0.45,0.40}%
\definecolor{mywhite}{rgb}{1,1,1}
\definecolor{myblack}{rgb}{0,0,0}

%%c++ newenvironment
\lstset{%
  language=C++,
  basicstyle= \fontsize{9}{1} \ttfamily,
  commentstyle=\itshape\color{darkgreen},
  keywordstyle=\bfseries\color{darkblue},
  stringstyle=\color{darkred},
  showspaces=false,
  showtabs=false,
  columns=fixed,
  %numbers=left,
  tabsize=3,
  %frame=trBL,
  backgroundcolor=\color{white},
  rulecolor=\color{shaderulecolor},
  %frame=single ,
  numberstyle=\tiny,
  breaklines=true,
  showstringspaces=false,
  xleftmargin=0cm,
  nolol=true,
  captionpos=b,
  morekeywords={Vector,VectorVector,EE,DglE1} }

\lstnewenvironment{cppcode}[1][]%
{
%\begin{center}
\minipage{\textwidth}
\renewcommand\lstlistingname{C++ Code Snipped}%
\renewcommand\lstlistlistingname{c++ Code Snippeds}%
\lstset{%
  language=C++,
  basicstyle= \fontsize{9}{9} \ttfamily,
  commentstyle=\bfseries\itshape\color{darkgreen},
  keywordstyle=\bfseries\color{darkblue},
  stringstyle=\color{darkred},
  showspaces=false,
  showtabs=false,
  columns=fixed,
  %numbers=left,
  tabsize=3,
  caption=#1,
  frame=tb,
  backgroundcolor=\color{bg},
  rulecolor=\color{black},
  %frame=single ,
  %numberstyle=\tiny,
  breaklines=true,
  showstringspaces=false,
  xleftmargin=0cm,
  nolol=true,
  captionpos=b,
  morekeywords={VectorVector,DglE1,EE,Model,Solver} }
}
{
\endminipage
%\end{center}
}

\newcommand{\parspace}{ $\;$\\  \\ }

\newcommand{\code}[1]{\lstinline| #1 |}

\begin{document}
    \begin{center}
        \huge % Schriftgröße einstellen
        \bfseries % Fettdruck einschalten
        \sffamily % Serifenlose Schrift
        Numerik Weihnachtsblatt\\[1em]
        \normalsize
        Kathrin Ronellenfitsch, Thorsten Beier, Christopher Pommrenke
    \end{center}

    \section*{Aufgabe 1}
    \subsection*{1.}
        Für die Differentialgleichung der Rentierpopulation gilt:
        \begin{equation*}
            u'(t) = 0.015u(t) - 3 * 10^{-5}u(t)^2 + 0.3
        \end{equation*}
    \subsection*{2.}
        Nach lösen mit RungeKutta4 erhält man zum Zeitpunkt $t = 358$ eine
        Rentierpopulation von $513$ Rentieren
        \begin{figure}[htbp]
            \centering
            \includegraphics{reindeerpopulation}%
            \caption{Verlauf der Rentierpopulation}%
        \end{figure}
        
        \newpage
        Implementierung des Rentiermodels:\\
        \begin{cppcode}
template<class T, class N=T>
class ReindeerModel
{
public:
  /** \brief export size_type */
  typedef std::size_t size_type;

  /** \brief export time_type */
  typedef T time_type;

  /** \brief export number_type */
  typedef N number_type;

  //! constructor stores parameter lambda
  ReindeerModel (const N& u_0_, const N& t_0_)
    : u_0(u_0_), t_0(t_0_)
  {}

  //! return number of componentes for the model
  std::size_t size () const
  {
    return 1;
  }

  //! set initial state including time value
  void initialize (T& t0, hdnum::Vector<N>& x0) const
  {
    t0 = t_0;//0;//t_0;
    x0[0] = u_0;//100;//u_0;
  }

  //! model evaluation
  void f (const T& t, const hdnum::Vector<N>& x, hdnum::Vector<N>& result) const
  {
    result[0] = 0.015 * x[0] - 0.00003 * x[0] * x[0] + 0.3;
  }

  //! jacobian evaluation needed for implicit solvers
  void f_x (const T& t, const hdnum::Vector<N>& x, hdnum::DenseMatrix<N>& result) const
  {
    throw("Jacobian evaluation not implemented");
  }

private:
  N u_0;
  N t_0;
};        
        \end{cppcode}
        
        \newpage
        Implementierung der main():\\
        \begin{cppcode}
int main () {
  // define a number type
  typedef double Number;
  // define constants
  const Number t0 = 0.0;
  const Number tMax = 358.0;
  const Number dt = 1.0;
  const Number u0 = 100.0;

  // model type
  typedef ReindeerModel<Number> Model;
  // instantiate model
  Model model(u0, t0);

  // storage for computed times and states
  Vector<Number> times;
  Vector<Vector<Number> > states;

  // solver type
  typedef RungeKutta4<Model> RKSolver;
  // instantiate solver
  RKSolver rksolver(model);
  rksolver.set_dt(dt);

  // set initial time and state
  times.push_back(rksolver.get_time());
  states.push_back(rksolver.get_state());

  // compute times and states
  Number t = t0;
  while(t < tMax){
    rksolver.step(); //make time step
    times.push_back(rksolver.get_time());
    states.push_back(rksolver.get_state());
    t = rksolver.get_time();
  }

  // write computed results in file
  gnuplot("reindeerpopulation.dat",times,states); // output model result
  std::cout << "reindeer population at time t = " << times.back() << ": " << states.back() << std::endl;
  return 0;
}
        
        \end{cppcode}
\end{document}