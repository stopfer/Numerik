%Schriftgröße, Layout, Papierformat, Art des Dokumentes
\documentclass[10pt,oneside,a4paper]{scrartcl}

%Einstellungen der Seitenränder
\usepackage[left=3cm,right=4cm,top=3cm,bottom=3cm,includeheadfoot]{geometry}

%neue Rechtschreibung
\usepackage{ngerman}

%Umlaute ermöglichen
\usepackage[latin1]{inputenc}

%Kopf- und Fußzeile
\usepackage{fancyhdr}
\pagestyle{fancy}
\fancyhf{}

%ermöglicht Text ein- / auszublenden
\usepackage{comment}

%\includecomment{Vortrag}
\excludecomment{Vortrag}

%to do
\usepackage{todonotes}

%Mathesachen
\usepackage{amsmath}
\usepackage{amssymb}

%Zeichnen
\usepackage{tikz}

%ermöglicht, Abbildung nicht zu benennen
\usepackage{caption}

%Linie oben
\renewcommand{\headrulewidth}{0.5pt}

%Linie unten
\renewcommand{\footrulewidth}{0.5pt}

%for captionof{figure}{"caption"}
\usepackage{caption}

%%for c++ syntax highlights
\usepackage{listings} 
\usepackage{verbatim}

%%for tikz
\usepackage{tikz,times}
\usetikzlibrary{mindmap,trees,backgrounds,arrows}
\tikzstyle{every picture}+=[remember picture]
\everymath{\displaystyle}

%%some colors
\definecolor{myblue}{RGB}{80,80,160}
\definecolor{mygreen}{RGB}{80,160,80}
\definecolor{darkblue}{rgb}{0,0,.6}
\definecolor{darkred}{rgb}{.6,0,0}
\definecolor{darkgreen}{rgb}{0,.5,0}
\definecolor{red}{rgb}{.98,0,0}
\definecolor{shellbackgroundcolor}{rgb}{1,1,1}
\definecolor{shellfontcolor}{rgb}{0,0,0}
\definecolor{shadethmcolor}{rgb}{1,1,1}
\definecolor{bg}{rgb}{0.975,0.98,1}
\definecolor{shaderulecolor}{rgb}{0.40,0.45,0.40}%
\definecolor{mywhite}{rgb}{1,1,1}
\definecolor{myblack}{rgb}{0,0,0}

%%c++ newenvironment
\lstset{%
  language=C++,
  basicstyle= \fontsize{9}{1} \ttfamily,
  commentstyle=\itshape\color{darkgreen},
  keywordstyle=\bfseries\color{darkblue},
  stringstyle=\color{darkred},
  showspaces=false,
  showtabs=false,
  columns=fixed,
  %numbers=left,
  tabsize=3,
  %frame=trBL,
  backgroundcolor=\color{white},
  rulecolor=\color{shaderulecolor},
  %frame=single ,
  numberstyle=\tiny,
  breaklines=true,
  showstringspaces=false,
  xleftmargin=0cm,
  nolol=true,
  captionpos=b,
  morekeywords={Vector,VectorVector,EE,DglE1} }

\lstnewenvironment{cppcode}[1][]%
{
%\begin{center}
\minipage{\textwidth}
\renewcommand\lstlistingname{C++ Code Snipped}%
\renewcommand\lstlistlistingname{c++ Code Snippeds}%
\lstset{%
  language=C++,
  basicstyle= \fontsize{9}{9} \ttfamily,
  commentstyle=\bfseries\itshape\color{darkgreen},
  keywordstyle=\bfseries\color{darkblue},
  stringstyle=\color{darkred},
  showspaces=false,
  showtabs=false,
  columns=fixed,
  %numbers=left,
  tabsize=3,
  caption=#1,
  frame=tb,
  backgroundcolor=\color{bg},
  rulecolor=\color{black},
  %frame=single ,
  %numberstyle=\tiny,
  breaklines=true,
  showstringspaces=false,
  xleftmargin=0cm,
  nolol=true,
  captionpos=b,
  morekeywords={VectorVector,DglE1,EE,Model,Solver} }
}
{
\endminipage
%\end{center}
}

\newcommand{\parspace}{ $\;$\\  \\ }

\newcommand{\code}[1]{\lstinline| #1 |}

\begin{document}
	%\title{Strukturen großer Netzwerke}
	%\author{Kathrin Ronellenfitsch}
	%\maketitle

	\begin{center}
		\huge % Schriftgröße einstellen
		\bfseries % Fettdruck einschalten
		\sffamily % Serifenlose Schrift
		Numerik Blatt 1\\[1em]
		\normalsize
		Kathrin Ronellenfitsch, Thorsten Beier, Christopher Pommrenke
	\end{center}

	
	\section*{Aufgabe 1}

    		\subsection*{1.1}

		\begin{align*}
			v'''''(x) - a(x)u'(x) &= f(x),	\\
			u''(x) - b(x)v(x) &= g(x)
		\end{align*}

		
		 \subsection*{1.2}

		\begin{align*}
			v'''''(x) - a(x)u''(x) &= f(x),	\\
			u''(x) - b(x)v(x) &= g(x)
		\end{align*}



	\section*{Aufgabe 2}

  		\subsection*{2.1}

		\begin{equation*}
			u'(t) = u(t)^2,\quad t \geq 0, \ u(0) = 1
		\end{equation*}
		
		\begin{equation*}
			f(t,u) = u =  u(t)^2
		\end{equation*}

		{\bf Eindeutigkeit:}
		\begin{equation*}
			||f(t,u) - f(t,v)|| =||u^2 - v^2|| = ||(u + v) (u - v)|| = ||u + v|| ||u - v|| \leq L(t) ||u - v||
		\end{equation*}
		mit $L(t) = \max\limits_{u,v ,\in D} ||u + v||  \Rightarrow$ die Funktion genuegt der lokalen Lipschitzbedingung $ \Rightarrow $ die Loesung der AWA ist eindeutig. 

 		\subsection*{2.2}

		\begin{equation*}
			u'(t) = -u(t)^2,\quad t \geq 0, \ u(0) = 1
		\end{equation*}

		\begin{equation*}
			f(t,u) = - u = - u(t)^2
		\end{equation*}

		{\bf Eindeutigkeit:}
		\begin{equation*}
			||f(t,u) - f(t,v)|| =||-u^2 + v^2|| = ||(v + u) (v - u)|| = ||v + u|| ||v - u|| = ||v + u|| ||-u - (-v)|| \leq L(t) || -u - (-v)|| 
		\end{equation*}
		mit $L(t) = \max\limits_{u,v ,\in D} ||u + v||  \Rightarrow$ die Funktion genuegt der lokalen Lipschitzbedingung $\Rightarrow$ die Loesung der AWA ist eindeutig. 



 		\subsection*{2.3}

		\begin{equation*}
			u'(t) = u(t)^{1/2},\quad t \geq 0, \ u(0) = 1
		\end{equation*}

		\begin{equation*}
			f(t,u) = u =  u(t)^{1/2}
		\end{equation*}

		{\bf Eindeutigkeit:}
		\begin{equation*}
			||f(t,u) - f(t,v)|| \leq L(t) ||u - v||  \Leftrightarrow \frac{||f(t,u) - f(t,v)||}{ ||u - v|| } \leq L(t)
		\end{equation*}
		
		Aber 
		\begin{equation*}
			 \frac{||f(t,u) - f(t,0)||}{ ||u - 0|| } = \left | \left | \frac{u^{1/2}}{u} \right | \right | =  \left | \left |  \frac{1}{\sqrt{u}} \right | \right | \to \infty \text { fuer } u \to 0 \quad \Rightarrow
		\end{equation*}

		\begin{equation*}
		\nexists L(t) \text { mit } \frac{||f(t,u) - f(t,v)||}{ ||u - v|| } \leq L(t)  \Leftrightarrow ||f(t,u) - f(t,v)|| \leq L(t) ||u - v|| \Rightarrow
		\end{equation*}

		die Funktion genuegt keiner Lipschitzbedingung


 		\subsection*{2.4}

		\begin{equation*}
			u'(t) = cos(u(t)) - 2u(t),\quad t \geq 0, \ u(0) = 1
		\end{equation*}

		\begin{equation*}
			 \frac{||f(t,u) - f(t,0)||}{ ||u - 0|| } = \left | \left | \frac{\cos(u) - 2u - \cos(0) - 2 \cdot 0}{u} \right| \right| =  \left | \left | \frac{\cos(u)}{u} - 2 - \frac{1}{u} \right| \right|
		\end{equation*}


\end{document}
